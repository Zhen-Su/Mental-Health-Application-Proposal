\hypertarget{requirements-analysis}{%
\section{Requirements Analysis}\label{requirements-analysis}}

\hypertarget{functional-requirements}{%
\subsection{Functional Requirements}\label{functional-requirements}}

\hypertarget{cloud-based}{%
\paragraph{1. Cloud-based}\label{cloud-based}}

\begin{itemize}
\tightlist
\item
  The local app will ensure the user will be able to retrieve their
  accounts and the contents from the cloud database onto a new device.

  \begin{itemize}
  \item
    The app must upload user data to server to sync user account
    information across the number of devices the user is using.
  \item
    The app will upload all user data to server

    \begin{itemize}
    \tightlist
    \item
      The app will upload questionnaire answers and preferences.
    \item
      The app will upload user-collected information.
    \item
      The app will upload API and login credential information.
    \end{itemize}
  \item
    The app will fetch user data from server
  \item
    The app will fetch data from connected APIs

    \begin{itemize}
    \tightlist
    \item
      The app will download new changes from the user's \textbf{Bham}
      calendar if given account credentials. (see: 6. Account
      Management)
    \item
      The app will download new changes from the user's \textbf{Canvas}
      account if given account credentials. (see: 6. Account Management)
    \end{itemize}
  \item
    The app will resolve changes between devices to prevent conflicts
    with information
  \item
    In case of unreachable to cloud server, will attempt upload once
    re-established connection.
  \end{itemize}
\end{itemize}

\hypertarget{registration-and-login}{%
\paragraph{2. Registration and Login}\label{registration-and-login}}

\begin{itemize}
\tightlist
\item
  The user will not be able to access the main functionalities of the
  app until they have either logged in or signed up for an account to
  log in.

  \begin{itemize}
  \tightlist
  \item
    The app will check if there is an account registered and log in,
    otherwise, the user will be directed to a log-in screen and has the
    option to create an account.

    \begin{itemize}
    \tightlist
    \item
      The user will log in with their credentials - email and password.
    \item
      Upon loggin in, the user will choose to save credentials for
      automatic login next time they start the app.
    \item
      ??? For security reasons, user credentials are stored in the cloud
      (see: 1. Cloud-based) as well as locally.
    \item
      ??? New password resets will be uploaded to the cloud, which is
      compared against locally saved password.
    \item
      The unchanged local password will fail from now on when being
      compared to new credentials on the cloud server.
    \end{itemize}
  \item
    When signing up for an account, the app will direct them to a new
    screen.

    \begin{itemize}
    \tightlist
    \item
      The user must provide the app with their credentials (email and
      password), and then a confirmation email will be sent to their
      email address in order to confirm the user-email.
    \item
      The user must confirm the email to verify registration.
    \item
      The user can provide the app with their Bham credentials and if
      done so:

      \begin{itemize}
      \tightlist
      \item
        The app will extract the information regarding schedule from
        their Bham account, and insert it into the calendar (see: 3.
        Calendar)
      \end{itemize}
    \item
      The user can provide the app with their Canvas credentials and if
      done so:

      \begin{itemize}
      \tightlist
      \item
        The app will extract information regarding deadlines and will
        add it to the calendar and todos (see: 3. Calendar, 4.Todo)
      \end{itemize}
    \item
      The user can provide their personal information through a
      questionnaire as part of the sign-up process. The user can choose
      to skip any of the given questions.

      \begin{itemize}
      \tightlist
      \item
        The questions from the questionnaire will be used to tailor the
        Behaviour Analysis AI output to the user needs. (see: 7. AI)
      \end{itemize}
    \item
      The user will be able to add or remove personal information in
      their account management after signup (see: 6. Account Management)
    \end{itemize}
  \item
    If the user has an account but has forgotten the password, they can
    opt to reset password by receiving a change password hyperlink to
    the email they used to register their account with.
  \end{itemize}
\end{itemize}

\hypertarget{calendar}{%
\paragraph{3. Calendar}\label{calendar}}

\begin{itemize}
\tightlist
\item
  The user will be able to modify the calendar upon signing into their
  account.

  \begin{itemize}
  \tightlist
  \item
    The user can add an event given all required information on name,
    date and length, with length 0 being a deadline event.
  \item
    The user can provide additional information on the event

    \begin{itemize}
    \tightlist
    \item
      location of event
    \item
      the pattern of the event of if it is a recurring event.
    \end{itemize}
  \item
    The user can delete an event, duplicate the event, or edit any
    sub-information belonging to the event.

    \begin{itemize}
    \tightlist
    \item
      the user will be able to specify if they are modifying for all
      subsequent events of the same name, or just the instance of the
      chosen event.
    \end{itemize}
  \end{itemize}
\item
  The calendar will be modified by fetching calendar information from
  Bham and Canvas if given access by user.

  \begin{itemize}
  \tightlist
  \item
    The app will have access to modification rights as if it was another
    user.
  \item
    The app will not consult user on adding events taken from the
    external sources, but the user is able to treat them as user-added
    events and modify them.
  \end{itemize}
\item
  The app will attempt to modify the calendar (the user can reject the
  changes) based on user-collected information from both the trackers
  (see: 5. Tracker) and personal information given from questionnaire
  (see: 6. Account Management) using the integrated AI (see: 7. AI).

  \begin{itemize}
  \tightlist
  \item
    The calendar will receive recommended output from AI based on given
    information to schedule the calendar with daily tasks (such as
    exercise, study etc.).
  \item
    The user can choose to accept none to all of the recommendation the
    AI makes.
  \item
    The AI will offer only 2 weeks worth of event recommendations
    because it needs to account for changes in the already existing
    schedule which can lead to uncertainty.
  \end{itemize}
\end{itemize}

\hypertarget{todo}{%
\paragraph{4. ToDo}\label{todo}}

\begin{itemize}
\tightlist
\item
  The todo acts as a daily representation of the calendar, and thus
  syncs information between itself and the Calendar.
\item
  It will display it in a phone-friendly format to allow user to better
  see task requirements for the day, in a focused manner.

  \begin{itemize}
  \tightlist
  \item
    It will add User events, Calendar events, and AI events and display
    them for the user based on current date.
  \end{itemize}
\item
  The user can create/edit/delete a todo in the same manner as they can
  an event in a calendar (see: 3.Calendar)

  \begin{itemize}
  \tightlist
  \item
    The new todo changes will be synced to calendar as an event.
  \end{itemize}
\end{itemize}

\hypertarget{tracker}{%
\paragraph{5. Tracker}\label{tracker}}

\begin{itemize}
\tightlist
\item
  The tracker will collect information from health trackers either
  built-in on device from applications, or external fitness trackers,
  and user input within the app if given.

  \begin{itemize}
  \tightlist
  \item
    The app will request permission to access existing tracked
    information on device and extract health information.
  \item
    The user can input tracking data themselves to the app.
  \item
    The app will display data for different time intervals about user
    health.
  \item
    The app will ask the user to give tracking data on their daily
    activities, such as sleep lengths.
  \item
    The user can choose to edit tracking data.
  \item
    The information collected through the tracker will be passed onto
    the AI (see: 7. AI) to create recommendations to improve the user's
    health.
  \item
    The tracker will send notifications to user about their health
    progress.
  \end{itemize}
\end{itemize}

\hypertarget{account-management}{%
\paragraph{6. Account Management}\label{account-management}}

\begin{itemize}
\tightlist
\item
  The user will be able to review their account information, to either
  add, edit, remove information.

  \begin{itemize}
  \tightlist
  \item
    The user can edit the questionnaire information that they provided
    when signing up.
  \item
    The user can edit their credentials, password and emails to change
    them.
  \item
    The user can see all the devices that are accessing the account.
  \item
    The user can log out of the app, which transfer to log-in screen.

    \begin{itemize}
    \tightlist
    \item
      The user can log out of all devices.
    \end{itemize}
  \end{itemize}
\item
  The user will be able to review and edit preferences regarding other
  functionalities of the app

  \begin{itemize}
  \tightlist
  \item
    The user can choose to turn on or off the AI (see: 7.Behaviour
    Analysis AI)
  \item
    The user can choose to delete AI stored information about user.
  \item
    The user can choose to sync cloud data manually.
  \item
    The user can choose to set sync period, or disable sync. it'll be
    set to a default value otherwise.
  \end{itemize}
\item
  The app will be able to make use of the information given by the user.
  The user must be able to set how much information the app can use.
\end{itemize}

\hypertarget{intelligent-behaviour-analysis-ai-iba}{%
\paragraph{7. Intelligent Behaviour Analysis AI
(IBA)}\label{intelligent-behaviour-analysis-ai-iba}}

\begin{itemize}
\tightlist
\item
  IBA will create scheduling models based on collected information from
  other functionalities of the app that the user can access

  \begin{itemize}
  \tightlist
  \item
    IBA collects information based on changes from the Calendar (see: 3.
    Calendar), User-given personal information (see: 6. Account
    Management), and from Tracker (see: 5. Tracker)
  \item
    IBA will attempt to sort the information coming in to produce an
    optimised schedule for a period of 2 weeks.
  \item
    IBA will recommend modifications to its own schedule model for the
    week based on actual user activity throughout the day collected from
    trackers (see: 5.Tracker).

    \begin{itemize}
    \tightlist
    \item
      i.e.~recommends more sleep hours for the following day if User
      reports light sleep for the previous day.
    \end{itemize}
  \item
    IBA will change its recommendation models based on how much of the
    models it recommends are rejected or accepted in attempt to study
    user preferences.
  \end{itemize}
\item
  IBA can be switched on or off by the user in Account Management (see:
  6. Account Management)
\item
  IBA's saved preferences can be deleted by the user in Profile
  Management (see: 6. Account Management)
\end{itemize}

\hypertarget{non-functional-requirements}{%
\subsection{Non Functional
Requirements}\label{non-functional-requirements}}

\hypertarget{to-add}{%
\paragraph{To add:}\label{to-add}}

\begin{itemize}
\tightlist
\item
  The app will upload user data to the server within 3 seconds
\item
  The app will fetch user data from server within 3 seconds (based on
  \textgreater1MB/s download speed)
\end{itemize}

\hypertarget{security}{%
\paragraph{1.Security}\label{security}}

\begin{itemize}
\tightlist
\item
  The app will implement hybrid cryptography for secure data transfer
  between cloud server and local app process used by user.

  \begin{itemize}
  \tightlist
  \item
    The app must ensure data communications with external sources are
    either local or also secure. Unless necessary, all communications
    are primarily between local app and cloud server.

    \begin{itemize}
    \tightlist
    \item
      Tracker applications and devices will be accessed without using
      any interaction between local app and cloud server.
    \item
      Tracker devices that rely on Bluetooth technology must also
      encrypt all transfer of communication between the device and the
      app.
    \item
      Fetching calendar/deadlines information from Bham API and Canvas
      API rely on the security of the respective systems.
    \item
      Credentials must be encrypted before transferring to API to avoid
      access-points into app or respective accounts of Bham and Canvas.
    \end{itemize}
  \end{itemize}
\end{itemize}

\hypertarget{reliability}{%
\paragraph{2.Reliability}\label{reliability}}

\begin{itemize}
\tightlist
\item
  The app should be designed to account for possible errors and failures
  in the components and attempt to address it without hurting user
  experience or minimising it.

  \begin{itemize}
  \tightlist
  \item
    The AI should be designed and stress tested in mind to handle any
    and all changes in the app components.
  \item
    The app should be designed to sync information from cloud coming
    from older versions of the app and updating the local app
    accordingly.
  \item
    Conversely, components within the local app must be designed so
    that, once newer versions of the app are introduced with changes to
    its components, it must be able to interpret information from older
    versions of app without error. The app will only modify information
    based on changes to component between versions.
  \end{itemize}
\item
  The cloud server should be designed with redundancy in mind, syncing
  with back-up server(s) after every set uploads from all users to
  account for possible failure from either servers.
\end{itemize}

\hypertarget{scalability}{%
\paragraph{3.Scalability}\label{scalability}}

\begin{itemize}
\tightlist
\item
  The app should be constructed that an increase of users does not
  adversely affect the user experience of the individual user.

  \begin{itemize}
  \tightlist
  \item
    Be designed that additional servers can be added or removed
    seamlessly and proportionally handle requests from users as user
    base increases or decreases.
  \item
    Load balancers control server traffic to prevent server overload.
    Cloud server maintains control over sync request from local app, and
    can issue earlier sync requests to reduce expected spikes in data
    due to set upload time, or divert sync request to additional
    identical servers.
  \end{itemize}
\end{itemize}

\hypertarget{efficiency}{%
\paragraph{4.Efficiency}\label{efficiency}}

\begin{itemize}
\tightlist
\item
  Actions the users may take must take a minimal amount of processing
  time of 1 second within the app.

  \begin{itemize}
  \tightlist
  \item
    Communications to external API i.e.~Canvas or Bham must not add on
    significant waiting time than based on the users internet connection
    speed.
  \item
    All background tasks should be done in the background and thus can
    take more time, but must finish as soon as possible

    \begin{itemize}
    \tightlist
    \item
      Server syncs should compare information between local app and
      cloud server to upload the smallest possible data.
    \item
      Behavioural Analysis AI should minimise time complexity given
      increase in information coming in to efficiently create
      recommendation models.

      \begin{itemize}
      \tightlist
      \item
        The BAI should compare incoming information with current one
        that it is actively using, and only pass new information to
        update recommendation models, rather than recreating them each
        time.
      \end{itemize}
    \end{itemize}
  \end{itemize}
\end{itemize}

\hypertarget{maintainability}{%
\paragraph{5.Maintainability}\label{maintainability}}

\begin{itemize}
\tightlist
\item
  The app should be able to operate with minimal human oversight

  \begin{itemize}
  \tightlist
  \item
    The BAI should be able to run autonomously without direct user
    interaction and run effectively with any given amount of information
    collected.
  \item
    The cloud servers should be able to operate either indecently from
    other servers in case of required server maintenance and updates.
  \item
    The cloud server should be set up to accept sync requests from older
    versions of the app as it only handles incoming packages of user
    information.
  \end{itemize}
\end{itemize}

\hypertarget{accessibility-and-usability}{%
\paragraph{6.Accessibility and
Usability}\label{accessibility-and-usability}}

\begin{itemize}
\tightlist
\item
  All of the app should be able to be used effectively with minimal
  instructions, and is intuitive to the user. The app should also be
  accessible by people with disabilities.

  \begin{itemize}
  \tightlist
  \item
    The design of the UI must take into account of colour blindness so
    that people with them are not confused.
  \item
    The questions asked by the questionnaire should be short,
    understandable, concise.
  \item
    General word descriptors of the app functionalities should be short,
    understandable, concise.
  \item
    Aim to display the app in mostly visuals instead of words.
  \end{itemize}
\end{itemize}
